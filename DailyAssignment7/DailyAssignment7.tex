%\documentclass[11pt,reqno]{amsart}
\documentclass[11pt,reqno]{article}
\usepackage[margin=.8in, paperwidth=8.5in, paperheight=11in]{geometry}
%\usepackage{geometry}                % See geometry.pdf to learn the layout options. There are lots.
%\geometry{letterpaper}                   % ... or a4paper or a5paper or ... 
%\geometry{landscape}                % Activate for for rotated page geometry
%\usepackage[parfill]{parskip}    % Activate to begin paragraphs with an empty line rather than an indent7
\usepackage{graphicx}
\usepackage{pstricks}
\usepackage{amssymb}
\usepackage{epstopdf}
\usepackage{amsmath}
\usepackage{subfigure}
\usepackage{caption}
\pagestyle{plain}
%\renewcommand{\topfraction}{0.3}
%\renewcommand{\bottomfraction}{0.8}
%\renewcommand{\textfraction}{0.07}
\DeclareGraphicsRule{.tif}{png}{.png}{`convert #1 `dirname #1`/`basename #1 .tif`.png}

\title{Real Analysis $\mathbb{II}$: \\ Daily Assignment 7}
\author{Andrew Rickert}
\date{Started: March 13, 2012 \\ \hspace{1pt} Ended: March ??,  2012}                                           % Activate to display a given date or no date

\begin{document}
\maketitle

% Page 1
\begin{flushleft} 
\textbf{Class 18.101} - Section 8 Problem 3\\
\rule{500pt}{1pt}\\
\end{flushleft} 

Let $f : \mathbb{R}^n \to \mathbb{R}^n$ be given by the equation $f(\textbf{x}) = ||\textbf{x}||^2 \cdot \textbf{x}$. Show that $f$ is of class $C^\infty$ and that $f$ carries the unit ball $B(0;1)$ onto itself in a one-to-one fashion. Show, however, that the inverse function is not differentiable at $\textbf{0}$.\\

The $i$-th component function of $f$ is $f_i(\textbf{x}) = (x_1^2 + x_2^2 + \cdots + x_n^2) x^i$. Since this is a sum of polynomials it is clear that $f_i$ is $C^\infty$. This is then true for $f$ as well since it is true for all the component functions.

To show that $f$ is onto we take $\textbf{x} \in B(0;1)$. This means that $||\textbf{x}|| < 1$. Now let \[ \textbf{x}' = \frac{\textbf{x}}{||\textbf{x}||^{\frac{2}{3}}} \]
Since $||\textbf{x}|| < 1 \implies ||\textbf{x}||^{\frac{1}{3}} < 1$ we have 
\[ ||\textbf{x}'|| = \frac{||\textbf{x}||}{||\textbf{x}||^{\frac{2}{3}}} = ||\textbf{x}||^{\frac{1}{3}} < 1 \implies \textbf{x}' \in B(0;1) \]
So for every value $\textbf{x} \in B(0;1)$ we have $\textbf{x}' \in B(0;1)$ and $f(\textbf{x}') = \textbf{x}$ since 
\[ f(\textbf{x}') = ||\textbf{x}'|| \cdot \textbf{x}' = || \frac{\textbf{x}}{||\textbf{x}||^{\frac{2}{3}}} ||^2 \frac{\textbf{x}}{||\textbf{x}||^{\frac{2}{3}}} =  \frac{||\textbf{x}||^2}{||\textbf{x}||^{\frac{4}{3}}} \frac{\textbf{x}}{||\textbf{x}||^{\frac{2}{3}}} = \textbf{x} \]
This shows that $f$ is onto.
To show that $f$ is one-to-one we note that 
\begin{align*}
& f(\textbf{x}_1) = f(\textbf{x}_2) \implies ||\textbf{x}_1||^2\cdot \textbf{x}_1 = ||\textbf{x}_2||^2\cdot \textbf{x}_2 \implies ||||\textbf{x}_1||^2\cdot \textbf{x}_1 || = ||||\textbf{x}_2||^2\cdot \textbf{x}_2|| \\
& \implies  ||\textbf{x}_1||^2\cdot ||\textbf{x}_1|| = ||\textbf{x}_2||^2\cdot ||\textbf{x}_2|| \implies ||\textbf{x}_1||^3 = ||\textbf{x}_2||^3 \implies ||\textbf{x}_1|| = ||\textbf{x}_2||
\end{align*}

\noindent Now returning to the expression $f(\textbf{x}_1) = f(\textbf{x}_2)$ we have 
\begin{eqnarray*} 
f(\textbf{x}_1) = f(\textbf{x}_2) &\implies& ||\textbf{x}||^2\cdot \textbf{x}_1 = ||\textbf{x}||^2\cdot \textbf{x}_2 \implies ||\textbf{x}||^2 (\textbf{x}_1-  \textbf{x}_2) = 0 \\
&\implies& \textbf{x}_1 -  \textbf{x}_2 = 0 \quad \text{since if $||\textbf{x}||$ = 0 the one-to-one relation is vacuously true}\\
&\implies& \textbf{x}_1 = \textbf{x}_2
\end{eqnarray*}
We have shown that $f(\textbf{x}_1) = f(\textbf{x}_2) \implies \textbf{x}_1 = \textbf{x}_2$ so $f$ is one-to-one.

%Looking over the demonstration that $f$ is onto suggests that the inverse function is
%\[ g(\textbf{y}) = \frac{\textbf{y}}{||\textbf{y}||^{\frac{2}{3}}} \]
%This can be verified if one calculates to show that $g(f(\textbf{x})) = \textbf{x}$. Since $f(\textbf{0}) = \textbf{0}$ we need to evaluate $D g(\textbf{0})$.

By theorem 7.4 in Munkres if we can show that $D f(\textbf{0})$ exists but that det$D f(\textbf{0}) = 0$ then $D g$ does not exist i.e. is not differentiable. 
Letting $\textbf{I}$ be the identity matrix the calculation for $D f$ gives
\[ D f(\textbf{x}) = ||\textbf{x}||^2 \textbf{I} + 2 \left[ \begin{array}{c}
x_1 \textbf{x} \\
x_2 \textbf{x} \\
\vdots \\
x_n \textbf{x} \\
\end{array} \right] \]
 
It is clear from this expression that both $D f(\textbf{0}) = \textbf{0}$ and therefore det $D f(\textbf{0}) = 0$.
 
\begin{flushleft} 
\textbf{Class 18.101} - Section 8 Problem 4\\
\rule{500pt}{1pt}\\
\end{flushleft} 

\noindent Let $g : \mathbb{R}^2 \to \mathbb{R}^2$ be given by the equation
\[ g(x,y) = (2 y e^{2x}, x e^y) \]
Let $f : \mathbb{R}^2 \to \mathbb{R}^3$ be given by the equation
\[ f(x,y) = (3x - y^2,2x + y,xy + y^3) \]

\noindent (a) Show that there is a neighborhood of $(0,1)$ that $g$ carries in a one-to-one fashion onto a neighborhood of $(0,2)$.\\

\noindent To show the existence of the neighborhood of $(0,1)$ for $g$ we only need to apply the inverse function theorem. The component functions of $g$ are products of $C^\infty$ functions so $g$ is $C^\infty$. The last part of the theorem to be verified is that $g$ is non-singular at $(0,1)$.\\
Calculating $D g$ gives
\[ D g = \left[ \begin{array}{cc}
4 y e^{2x} & 2 e^{2x} \\
e^y & x e^y \\
\end{array} \right] \]
\noindent at $(0,1)$ we have
\[ D g(0,1) = \left[ \begin{array}{cc}
4 & 2 \\
e & 0 \\
\end{array} \right] \]
since det $D g(0,1) = -2e$ the function $g$ is non-singular at $(0,1)$ and the neighborhood exists.\\

\noindent (b) Find $D(f \circ g^{-1})$ at $(2,0)$.\\

The component functions of $f$ are products, sums and differences of $C^\infty$ functions so $f$ must also be $C^\infty$. By the chain rule and theorem 7.4 of Munkres we can simply the derivative calculation to
\[D(f \circ g^{-1}) (2,0) = D f (g^{-1} (2,0)) \cdot D g^{-1} (2,0) = D(f(0,1)) \cdot D g^{-1} (2,0) = D(f(0,1)) \cdot [D g (0,1)]^{-1} \]

\noindent To complete the calculation we need to calculate $D f(0,1)$ so
\[ D f(x,y) = \left[ \begin{array}{cc}
3 & -2 y \\
2 & 1 \\
y & x + 3 y^2 \\
\end{array} \right]  \implies D f(0,1) = \left[ \begin{array}{cc}
3 & -2 \\
2 & 1 \\
1 & 3 \\
\end{array} \right] \]

\noindent Using part $a$ we have for $[D g (0,1)]^{-1}$
\[
[D g (0,1)]^{-1} = \frac{1}{2 e} \left[ \begin{array}{cc}
0 & 2 \\
e & -4 \\
\end{array} \right] \]

\noindent This gives
\[ D(f \circ g^{-1}) (2,0) = D(f(0,1)) \cdot [D g (0,1)]^{-1} =   \left[ \begin{array}{cc}
3 & -2 \\
2 & 1 \\
1 & 3 \\
\end{array} \right] \cdot \frac{1}{2 e} \left[ \begin{array}{cc}
0 & 2 \\
e & -4 \\
\end{array} \right] =  \frac{1}{2 e} \left[ \begin{array}{cc}
-2e & 14 \\
e & 0 \\
3 e & -10 \\
\end{array} \right] \]


\begin{flushleft} 
\textbf{Class 18.101} - Section 8 Problem 5\\
\rule{500pt}{1pt}\\
\end{flushleft} 

Let $A$ be open in $\mathbb{R}^n$; let $f : A \to \mathbb{R}^n$ be of class $C^r$; assume $Df(\textbf{x})$ is non-singular for $\textbf{x} \in A$. Show that even if $f$ is not one-to-one on $A$, the set $B = f(A)$ is open in $\mathbb{R}^n$.

\end{document}
