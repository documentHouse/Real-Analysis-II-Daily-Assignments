%\documentclass[11pt,reqno]{amsart}
\documentclass[11pt,reqno]{article}
\usepackage[margin=.8in, paperwidth=8.5in, paperheight=11in]{geometry}
%\usepackage{geometry}                % See geometry.pdf to learn the layout options. There are lots.
%\geometry{letterpaper}                   % ... or a4paper or a5paper or ... 
%\geometry{landscape}                % Activate for for rotated page geometry
%\usepackage[parfill]{parskip}    % Activate to begin paragraphs with an empty line rather than an indent7
\usepackage{graphicx}
\usepackage{pstricks}
\usepackage{amssymb}
\usepackage{epstopdf}
\usepackage{amsmath}
\usepackage{subfigure}
\usepackage{caption}
\pagestyle{plain}
%\renewcommand{\topfraction}{0.3}
%\renewcommand{\bottomfraction}{0.8}
%\renewcommand{\textfraction}{0.07}
\DeclareGraphicsRule{.tif}{png}{.png}{`convert #1 `dirname #1`/`basename #1 .tif`.png}

\title{Real Analysis $\mathbb{II}$: \\ Daily Assignment 6}
\author{Andrew Rickert}
\date{Started: March 7, 2012 \\ \hspace{1pt} Ended: March 12,  2012}                                           % Activate to display a given date or no date

\begin{document}
\maketitle

% Page 1
\begin{flushleft} 
\textbf{Class 18.101} - Section 8 Problem 1\\
\rule{500pt}{1pt}\\
\end{flushleft} 

\noindent Let $f : \mathbb{R}^2 \to \mathbb{R}^2$ be defined by the equation

\[ f(x,y) = (x^2 - y^2, 2 x y) \]

\noindent (a) Show that $f$ is one-to-one on the set $A$ consisting of all $(x,y)$ with $x > 0$. \\

Suppose that $f$ is not one to one. There must exist two points $\textbf{a}_0 = (x_0,y_0)$ and $\textbf{a}_1 = (x_1,y_1)$ such that $\textbf{a}_0 \neq \textbf{a}_1$ but $f(\textbf{a}_0) = f(\textbf{a}_1)$. Now define two functions as follows:
\begin{eqnarray*}
\phi(x,y) &=& ||f(x,y)||\\
d(t) &=& \phi(\textbf{a}_0 + t(\textbf{a}_1 - \textbf{a}_0))\\
\end{eqnarray*}

\noindent We can apply the mean value theorem to $d(t)$ with $0 = d(0) = d(1)$ to get a value $t_0$ such that 
\[ 0 = d(1) - d(0) = d'(t_0)(1 - 0) = d'(t_0) = D \phi(\textbf{a}_0 + t_0(\textbf{a}_1 - \textbf{a}_0)) \cdot (\textbf{a}_1 - \textbf{a}_0) \]
Since $\textbf{a}_1 - \textbf{a}_0 \neq \textbf{0}$ by assumption so we have $D \phi(\textbf{a}_0 + t_0(\textbf{a}_1 - \textbf{a}_0)) = 0$. If we can show that $D \phi(x,y)$ is not zero for any $(x,y) \in A$ then we contradict that $f(x,y)$ is \emph{not} one-to-one.

 \begin{eqnarray*} 
 D \phi(x,y) &=& \left[ \begin{array}{cc} 
 D_1 \sqrt{(x^2 - y^2)^2 + (2 x y)^2} & D_2 \sqrt{(x^2 - y^2)^2 + (2 x y)^2}\\
\end{array} \right] \\
		 &=& \left[ \begin{array}{cc} 
 \frac{x(x^2 - y^2) + 2 x y^2}{\sqrt{(x^2 - y^2)^2 + (2 x y)^2}} & \frac{-y(x^2 - y^2) + 2 x^2 y}{\sqrt{(x^2 - y^2)^2 + (2 x y)^2}}\\
\end{array} \right] \\
\end{eqnarray*}

Since we are trying to find possible $(x,y)$ where $D \phi(x,y) = 0$ we set the above matrix to zero and derive the following equations

\begin{eqnarray*}
\frac{x(x^2 - y^2) + 2 x y^2}{\sqrt{(x^2 - y^2)^2 + (2 x y)^2}} &=& 0\\
\frac{-y(x^2 - y^2) + 2 x^2 y}{\sqrt{(x^2 - y^2)^2 + (2 x y)^2}} &=& 0\\
\end{eqnarray*}

\noindent But since $0 < x$ we know that $\sqrt{(x^2 - y^2)^2 + (2 x y)^2} \neq 0$ so 

\begin{align}
&\hspace{38pt}x(x^2 - y^2) + 2 x y^2 = 0 \label{eqn:deriv1}\\
&\hspace{25pt}-y(x^2 - y^2) + 2 x^2 y = 0 \label{eqn:deriv2}\\
&\implies x^2 (x^2 - y^2) + y^2 (x^2 - y^2 ) = 0 \nonumber \\
&\implies (x^2 + y^2) (x^2 - y^2) = 0 \nonumber
\end{align}

Since $0 < x$ we know that $x^2 + y^2 \neq 0$ we have that $x^2 - y^2 = 0$. This implies that $y = \pm x$ but since $0 < x$ it must be the case that $x = y$.\\
Looking at equations $(\ref{eqn:deriv1})$ and $(\ref{eqn:deriv2})$ again we can also derive

\begin{eqnarray*} 
x y^3 + x^3 y &=& 0 \\
y y^3 + y^3 y &=& 2 y^4 = 0 \quad \text{since $x = y$}\\
\implies y &=& 0\\
\implies x &=& 0 \quad \text{since $x = y$}
\end{eqnarray*}

Thus the only value where $D \phi(x,y) = 0$ is when $(x,y) = 0$. Since there are no values in $A$ for which $D \phi(x,y) = 0$ we have the necessary contradiction.\\

\noindent (b) What is the set $B = f(A)$?\\

\noindent Since $f(x,y) = (x^2 - y^2, 2 x y)$ we can let $u = x^2 - y^2$ and $v = 2 x y$. This gives
\begin{align*}
&u^2 = x^4 - 2 x^2 y^2 + y^4, \quad v^2 = 4 x^2 y^2 \\
&\implies u^2  + \frac{v^2}{2} = x^4 + y^4\\
\end{align*}

If we let a constant $a$ be such that $x^4 + y^4 = a^2$ then we have the ellipse in $(u,v)$ space:
\[ u^2  + \frac{v^2}{2} = a^2 \]
 
Since $u(-x,-y) = u(-x,y) = u(x,y)$, $v(-x,-y) = v(x,y)$ and $v(-x,y) = v(x,y)$ it is clear that the restriction of $0 < x$ will not restrict the range of $(u,v)$ in the ellipse. There is one exception. Eventhough we may replace a $(-x,y)$ with a $(x,y)$ to produce the value that would have been absent because we do not allow $x < 0$ there is not replacement for $x = 0$. In the case $x = 0$ it is true that $(u,v) = (-y^2,0)$. This means that all points on the line $(\infty,0] \times 0$ are excluded from $(u,v)$ space. The ellipses are all disjoint and partition $\mathbb{R}^2$ so the mapped set is 
\[ B = \mathbb{R} - (\infty,0] \times 0 \]

\noindent (c) If $g$ is the inverse function, $D g(0,1)$\\

First we note that $f(\frac{1}{\sqrt{2}},\frac{1}{\sqrt{2}}) = (0,1)$ so that $g(0,1) = (\frac{1}{\sqrt{2}},\frac{1}{\sqrt{2}})$. Because $(\frac{1}{\sqrt{2}},\frac{1}{\sqrt{2}}) \in A$ and $A$ is open we know by the inverse function theorem that the inverse function exists and is differentiable at $(0,1)$. By theorem 7.4 we may calculate the derivative as
\[ D g(0,1) = D f(\frac{1}{\sqrt{2}},\frac{1}{\sqrt{2}})^{-1} \]

\noindent So we calculate $D f$
\[ D f(x,y) = \left[ \begin{array}{cc}
2x & -2y \\
2y & 2x \\
\end{array} \right] \implies D f(x,y)^{-1} = \frac{1}{4(x^2 + y^2)} \left[ \begin{array}{cc}
2x & 2y \\
-2y & 2x \\
\end{array} \right] \]

\noindent This gives 
\[ D g(0,1) = D f(\frac{1}{\sqrt{2}},\frac{1}{\sqrt{2}})^{-1} =  \left[ \begin{array}{cc}
\frac{1}{2 \sqrt{2}} & \frac{1}{2 \sqrt{2}}  \\
-\frac{1}{2 \sqrt{2}}  & \frac{1}{2 \sqrt{2}}  \\
\end{array} \right] \]

\begin{flushleft} 
\textbf{Class 18.101} - Section 8 Problem 2\\
\rule{500pt}{1pt}\\
\end{flushleft} 

\noindent Let $f : \mathbb{R}^2 \to \mathbb{R}^2$ be defined by the equation

\[ f(x,y) = (e^x \cos y, e^x \sin y) \]

\noindent (a) Show that $f$ is one-to-one on the set $A$ consisting of all $(x,y)$ with $x > 0$. \\

Using the same argument from part $a$ of the previous problem, we can show that $f$ is one-to-one by contradiction if we show that $D phi \neq 0$ for all $(x,y) \in A$ where 
\[ \phi(x,y) = \sqrt{ (e^x \cos y)^2 + (e^x \sin y)^2 } = \sqrt{e^{2x}} = e^x \]
\noindent Calculating $D \phi(x,y)$ gives

\[ D \phi(x,y) = \left[ \begin{array}{cc}
e^x & 0 \\
\end{array} \right] \]

Since $e^x \neq 0$ for all $x$ we have shown by the argument from the previous problem that $f$ is one-to-one.\\

\noindent (b) What is the set $B = f(A)$?\\

If we let $u = e^x \cos y$ and $v = e^x \sin y$ then we may derive the relation
\[ u^2 + v^2 = (e^x \cos y)^2 + (e^x \sin y)^2 = e^{2x} \]

This defines a collection of circles in $(u,v)$ space of radius $e^x$. Because $0 < y < 2 \pi$ we know that the circles will not include that values at $y = 0$. By the definition of $f$ these will be the set $(e^x,0)$ for $x \in \mathbb{R}$ which is $(0,\infty) \times 0$. It is also clear from the equation describing the circles that $(0,0) \notin B$ which gives the result 
\[ B = \mathbb{R} - [0,\infty) \times 0 \]

\noindent (c) If $g$ is the inverse function, $D g(0,1)$\\

First we note that $f(0,\frac{\pi}{2}) = (0,1)$ so that $g(0,1) = (0,\frac{\pi}{2})$. Because $(0,\frac{\pi}{2}) \in A$ and $A$ is open we know by the inverse function theorem that the inverse function exists and is differentiable at $(0,1)$. By theorem 7.4 we may calculate the derivative as
\[ D g(0,1) = D f(\frac{1}{\sqrt{2}},\frac{1}{\sqrt{2}})^{-1} \]


\noindent So we calculate $D f$
\[ D f(x,y) = \left[ \begin{array}{cc}
e^x \cos y & - e^x \sin y \\
e^x \sin y & e^x \cos y \\
\end{array} \right] \implies D f(x,y)^{-1} = \frac{1}{e^{2x}} \left[ \begin{array}{cc}
e^x \cos y & e^x \sin y \\
-e^x \sin y & e^x \cos y \\
\end{array} \right] \]

\noindent This gives 
\[ D g(0,1) = D f(0,\frac{\pi}{2})^{-1} =  \left[ \begin{array}{cc}
0 & 1 \\
-1 & 0 \\
\end{array} \right] \]

\end{document}
