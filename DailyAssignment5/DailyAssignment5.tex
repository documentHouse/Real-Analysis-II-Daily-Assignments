%\documentclass[11pt,reqno]{amsart}
\documentclass[11pt,reqno]{article}
\usepackage[margin=.8in, paperwidth=8.5in, paperheight=11in]{geometry}
%\usepackage{geometry}                % See geometry.pdf to learn the layout options. There are lots.
%\geometry{letterpaper}                   % ... or a4paper or a5paper or ... 
%\geometry{landscape}                % Activate for for rotated page geometry
%\usepackage[parfill]{parskip}    % Activate to begin paragraphs with an empty line rather than an indent7
\usepackage{graphicx}
\usepackage{pstricks}
\usepackage{amssymb}
\usepackage{epstopdf}
\usepackage{amsmath}
\usepackage{subfigure}
\usepackage{caption}
\pagestyle{plain}
%\renewcommand{\topfraction}{0.3}
%\renewcommand{\bottomfraction}{0.8}
%\renewcommand{\textfraction}{0.07}
\DeclareGraphicsRule{.tif}{png}{.png}{`convert #1 `dirname #1`/`basename #1 .tif`.png}

\title{Real Analysis $\mathbb{II}$: \\ Daily Assignment 5}
\author{Andrew Rickert}
\date{Started: March 5, 2012 \\ \hspace{1pt} Ended: March 5,  2012}                                           % Activate to display a given date or no date

\begin{document}
\maketitle

% Page 1
\begin{flushleft} 
\textbf{Class 18.101} - Section 7 Problem 1\\
\rule{500pt}{1pt}\\
\end{flushleft} 

\noindent Let $f : \mathbb{R}^3 \to \mathbb{R}^2$ satisfy the conditions $f(\textbf{0}) = (1,2)$ and 
\[ D f(\textbf{0}) = \left[ \begin{array}{ccc}
1 & 2 & 3 \\
0 & 0 & 1\\
\end{array} \right] \]

\noindent Let $g : \mathbb{R}^2 \to \mathbb{R}^2$ be defined by the equation
\[ g(x,y) = (x + 2 y, 3 x y) \]
\noindent Find $D(g \circ f)(\textbf{0})$.\\

\noindent By the chainrule and the statement of the problem we have 
\[ D(g \circ f)(\textbf{0})  =  D g(f(\textbf{0})) \cdot D f(\textbf{0}) =  D g(1,2) \cdot D f(\textbf{0}) \]

The value of $D f(\textbf{0})$ is already given to us. We only need to find $D g$. Since the partials of $g$ are clearly continuous we know that the derivative of $g$ will be given by:
\[ D g(x,y) = \left[ \begin{array}{cc}
D_1 g_1 & D_2 g_1 \\
D_1 g_2 & D_2 g_2 \\
\end{array} \right]  = \left[ \begin{array}{cc}
1 & 2 \\
3 y & 3 x \\
\end{array} \right] \]

\noindent So we get 
\[ D g(1,2) =  \left[ \begin{array}{cc}
1 & 2 \\
6 & 3 \\
\end{array} \right] \]

\noindent We derive the answer
\[ D(g \circ f)(\textbf{0})  = D g(f(\textbf{0})) \cdot D f(\textbf{0})  = \left[ \begin{array}{cc}
1 & 2 \\
6 & 3 \\
\end{array} \right] \cdot \left[ \begin{array}{ccc}
1 & 2 & 3 \\
0 & 0 & 1\\
\end{array} \right] = \left[ \begin{array}{ccc}
1 & 2 & 5 \\
6 & 12 & 21\\
\end{array} \right]\] 
\newpage

\begin{flushleft} 
\textbf{Class 18.101} - Section 7 Problem 2\\
\rule{500pt}{1pt}\\
\end{flushleft} 

\noindent Let $f : \mathbb{R}^2 \to \mathbb{R}^3$ and $g : \mathbb{R}^3 \to \mathbb{R}^2$ be given by the equations
\begin{eqnarray*}
f(\textbf{x}) &=& (e^{2 x_1 + x_2}, 3 x_2 - \cos \, x_1, x_1^2 + x_2 + 2) \\
g(\textbf{y}) &=& (3 y_1 + 2 y_2 + y_3^2, y_1^2 - y_3 + 1) \\
\end{eqnarray*}

\noindent (a) $F(\textbf{x}) = g(f(\textbf{x}))$, find $DF(\textbf{0})$. \\

\noindent First we note that $f(\textbf{0}) = (1,-1,2)$ and by the chain rule we have:

\[ D(g \circ f)(\textbf{0})  =  D g(f(\textbf{0})) \cdot D f(\textbf{0}) =  D g(1,-1,2) \cdot D f(0,0) \]

\noindent To complete the problem we need to calculate the derivatives of the functions individually.

\begin{eqnarray*}
D f(x_1,x_2) &=& \left[ \begin{array}{cc}
D_1 f_1 & D_2 f_1\\
D_1 f_2 & D_2 f_2 \\
D_1 f_3 & D_2 f_3 \\
\end{array} \right]  = \left[ \begin{array}{cc}
2 e^{2 x_1 + x_2} & e^{2 x_1 + x_2} \\
\sin \, x_1 & 3 \\
2 x_1 & 1 \\
\end{array} \right] \\
D g(y_1,y_2,y_3) &=& \left[ \begin{array}{ccc}
D_1 g_1 & D_2 g_1 & D_3 g_1\\
D_1 g_2 & D_2 g_2 & D_3 g_3 \\
\end{array} \right]  = \left[ \begin{array}{ccc}
3 & 2 & 2 y_3\\
2 y_1 & 0 & -1\\
\end{array} \right]
\end{eqnarray*}

\noindent So we have
\[   D g(1,-1,2) \cdot D f(0,0) =   \left[ \begin{array}{ccc}
3 & 2 & 4\\
2 & 0 & -1\\
\end{array} \right] \cdot \left[ \begin{array}{cc}
2 & 1 \\
0 & 3 \\
0 & 1 \\
\end{array} \right] = \left[ \begin{array}{cc}
6 & 13 \\
4 & 1 \\
\end{array} \right] \] \\

\begin{flushleft} 
\textbf{Class 18.101} - Section 7 Problem 3\\
\rule{500pt}{1pt}\\
\end{flushleft} 

\noindent Let $f : \mathbb{R}^3 \to \mathbb{R}$ and $g : \mathbb{R}^2 \to \mathbb{R}$ be differentiable. Let $F : \mathbb{R}^2 \to \mathbb{R}$ by defined by the equation
\[ F(x,y) = f(x,y,g(x,y)) \] 

\noindent (a) Find $DF$ in terms of the partials of $f$ and $g$\\

\noindent We rewrite $F$ as a composition $F = c \circ e$ where 
\begin{eqnarray*}
e(x,y) &=& (x,y,g(x,y)) \\
c(x,y,z) &=& f(x,y,z) \\
\end{eqnarray*}

\noindent We can now calculate $D F$ as follows:
\[ D F(x,y) = D (c \circ e) (x,y) =  D c(e(x,y)) \cdot D e(x,y) \]

\noindent The derivatives are as follows:
\[ De = \left[ \begin{array}{cc}
D_1 e_1 & D_2 e_1 \\
D_1 e_2 & D_2 e_2 \\
D_1 e_3 & D_2 e_3 \\
\end{array} \right] =  \left[ \begin{array}{cc}
1 & 0 \\
0 & 1 \\
D_1 g(x,y) & D_2 g(x,y) \\
\end{array} \right] \]

\noindent and
\[ Dc = \left[ \begin{array}{ccc}
D_1 f_1 & D_2 f_1 & D_3 f_1\\
\end{array} \right] =  \left[ \begin{array}{ccc}
D_1 f(x,y,g(x,y)) & D_2 f(x,y,g(x,y)) & D_3 f(x,y,g(x,y)) \\
\end{array} \right] \]

\noindent This gives a value for $DF$
\begin{eqnarray*}
D F(x,y) &=& D c(e(x,y)) \cdot D e(x,y)\\ 
&=& \left[ \begin{array}{ccc}
D_1 f(x,y,g(x,y)) & D_2 f(x,y,g(x,y)) & D_3 f(x,y,g(x,y)) \\
\end{array} \right] \cdot \left[ \begin{array}{cc}
1 & 0 \\
0 & 1 \\
D_1 g(x,y) & D_2 g(x,y) \\
\end{array} \right] \\
&=&  \left[ \begin{array}{c}
D_1 f(x,y,g(x,y)) + D_3 f(x,y,g(x,y)) \cdot D_1 g(x,y) \\
 D_2 f(x,y,g(x,y)) + D_3 f(x,y,g(x,y))\cdot D_2 g(x,y) \\
\end{array} \right]
\end{eqnarray*} \\

\noindent (b) If $F(x,y) = 0$ for all (x,y), find $D_1 g$ and $D_2 g$ in terms of the partials of $f$. \\

\noindent If $F(x,y) = 0$ then $D F(x,y) = 0$, from part $a$ we have then that 
\[ \left[ \begin{array}{c}
D_1 f(x,y,g(x,y)) + D_3 f(x,y,g(x,y)) \cdot D_1 g(x,y) \\
D_2 f(x,y,g(x,y)) + D_3 f(x,y,g(x,y))\cdot D_2 g(x,y) \\
\end{array} \right]  = \left[ \begin{array}{c} 
0 \\
0 \\
\end{array} \right] \]

\noindent This shows that 
\[
\begin{array}{c}
0 = D_1 f(x,y,g(x,y)) + D_3 f(x,y,g(x,y)) \cdot D_1 g(x,y) \\
0 = D_2 f(x,y,g(x,y)) + D_3 f(x,y,g(x,y)) \cdot D_2 g(x,y) \\
\end{array}
\implies \begin{array}{c}
 D_1 g(x,y) = \frac{-D_1 f(x,y,g(x,y))}{D_3 f(x,y,g(x,y))} \\
D_2 g(x,y) =  \frac{-D_2 f(x,y,g(x,y))}{D_3 f(x,y,g(x,y))}  \\
\end{array}
\]

\end{document}
