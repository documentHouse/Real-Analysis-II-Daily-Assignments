%\documentclass[11pt,reqno]{amsart}
\documentclass[11pt,reqno]{article}
\usepackage[margin=.8in, paperwidth=8.5in, paperheight=11in]{geometry}
%\usepackage{geometry}                % See geometry.pdf to learn the layout options. There are lots.
%\geometry{letterpaper}                   % ... or a4paper or a5paper or ... 
%\geometry{landscape}                % Activate for for rotated page geometry
%\usepackage[parfill]{parskip}    % Activate to begin paragraphs with an empty line rather than an indent7
\usepackage{graphicx}
\usepackage{pstricks}
\usepackage{amssymb}
\usepackage{epstopdf}
\usepackage{amsmath}
\usepackage{subfigure}
\usepackage{caption}
\pagestyle{plain}
%\renewcommand{\topfraction}{0.3}
%\renewcommand{\bottomfraction}{0.8}
%\renewcommand{\textfraction}{0.07}
\DeclareGraphicsRule{.tif}{png}{.png}{`convert #1 `dirname #1`/`basename #1 .tif`.png}

\title{Real Analysis $\mathbb{II}$: \\ Daily Assignment 5}
\author{Andrew Rickert}
\date{Started: March 5, 2012 \\ \hspace{1pt} Ended: March ??,  2012}                                           % Activate to display a given date or no date

\begin{document}
\maketitle

% Page 1
\begin{flushleft} 
\textbf{Class 18.101} - Section 7 Problem 1\\
\rule{500pt}{1pt}\\
\end{flushleft} 

\noindent Let $f : \mathbb{R}^3 \to \mathbb{R}^2$ satisfy the conditions $f(\textbf{0}) = (1,2)$ and 
\[ D f(\textbf{0}) = \left[ \begin{array}{ccc}
1 & 2 & 3 \\
0 & 0 & 1\\
\end{array} \right] \]

\noindent Let $g : \mathbb{R}^2 \to \mathbb{R}^2$ be defined by the equation
\[ g(x,y) = (x + 2 y, 3 x y) \]
\noindent Find $D(g \circ f)(\textbf{0})$.

\begin{flushleft} 
\textbf{Class 18.101} - Section 7 Problem 2\\
\rule{500pt}{1pt}\\
\end{flushleft} 



\begin{flushleft} 
\textbf{Class 18.101} - Section 7 Problem 3\\
\rule{500pt}{1pt}\\
\end{flushleft} 



\end{document}
