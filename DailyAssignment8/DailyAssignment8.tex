%\documentclass[11pt,reqno]{amsart}
\documentclass[11pt,reqno]{article}
\usepackage[margin=.8in, paperwidth=8.5in, paperheight=11in]{geometry}
%\usepackage{geometry}                % See geometry.pdf to learn the layout options. There are lots.
%\geometry{letterpaper}                   % ... or a4paper or a5paper or ... 
%\geometry{landscape}                % Activate for for rotated page geometry
%\usepackage[parfill]{parskip}    % Activate to begin paragraphs with an empty line rather than an indent7
\usepackage{graphicx}
\usepackage{pstricks}
\usepackage{amssymb}
\usepackage{epstopdf}
\usepackage{amsmath}
\usepackage{subfigure}
\usepackage{caption}
\pagestyle{plain}

\def\uint{\mathchoice%
    {\mkern13mu\overline{\vphantom{\intop}\mkern9mu}\mkern-20mu}%
    {\mkern9mu\overline{\vphantom{\intop}\mkern9mu}\mkern-14mu}%
    {\mkern9mu\overline{\vphantom{\intop}\mkern9mu}\mkern-14mu}%
    {\mkern9mu\overline{\vphantom{\intop}\mkern9mu}\mkern-14mu}%
  \int}
\def\lint{\mkern3mu\underline{\vphantom{\intop}\mkern9mu}\mkern-10mu\int}

\DeclareGraphicsRule{.tif}{png}{.png}{`convert #1 `dirname #1`/`basename #1 .tif`.png}

\title{Real Analysis $\mathbb{II}$: \\ Daily Assignment 8}
\author{Andrew Rickert}
\date{Started: March 21, 2012 \\ \hspace{1pt} Ended: March 23,  2012}                                           % Activate to display a given date or no date

\begin{document}
\maketitle


% Page 1
\begin{flushleft} 
\textbf{Class 18.101} - Section 10 Problem 1\\
\rule{500pt}{1pt}\\
\end{flushleft} 

Let $f,g : Q \to \mathbb{R}$ be bounded functions such that $f(\textbf{x}) \le g(\textbf{x})$ for $\textbf{x} \in Q$. \\
Show that $\lint_Q f \le \lint_Q g$ and $\uint_Q f \le \uint_Q g$.\\

We will prove the first part $\lint_Q f \le \lint_Q g$ since the proof of the second part is analogous.\\
We start by quickly showing that $f(\textbf{x}) \le g(\textbf{x}) \implies \text{inf} \{ f(\textbf{x}) \} \le  \text{inf} \{ g(\textbf{x}) \}$.

Let $F = \text{inf} \{ f(\textbf{x}) \}$ and $G = \text{inf} \{ g(\textbf{x}) \}$. Since $F$ is a lower bound we have $F \le f(\textbf{x}) \le g(\textbf{x})$ for all $\textbf{x}$ in the domain so $F$ is a lower bound for $\{ g(\textbf{x}) \}$. Because $G$ is the \emph{greatest} lower bound for $\{ g(\textbf{x}) \}$ we know that for any other bound $B$ of $\{ g(\textbf{x}) \}$ we have the relation $B \le G$. Since $F$ is such a lower bound we have $F \le G$ or \[ \text{inf} \{ f(\textbf{x}) \} \le  \text{inf} \{ g(\textbf{x}) \} \]

This allows us to say that $m_R(f) = \text{inf} \{ f(\textbf{x} | \textbf{x} \in R)\} \le \text{inf} \{ g(\textbf{x} | \textbf{x} \in R)\} = m_R(g)$.

Now we show that for any partition $P$ it is the case that $f(\textbf{x}) \le g(\textbf{x}) \implies L(f,P) \le L(g,P)$. The following calculation shows this 

\[
L(f,P) = \sum_R m_R(f) \cdot \nu (R) \le \sum_R m_R(g) \cdot \nu(R)= L(g,R)
\]

We can now prove the result. Because $\lint_Q f \equiv \text{sup} \: L(f,P)$ for any given $\epsilon$ there must be a partition $P$ such that 
\[ \lint_Q f - \epsilon < L(f,P) \]

\noindent We then have from the above comments
\begin{eqnarray*} 
L(f,P) &\le& \hspace{-10pt} L(g,P) \le \lint_Q g \quad \text{since the lower integral is sup $L(g,P)$} \\
	 &\implies& \lint_Q f - \epsilon < \lint_Q g \implies \lint_Q f < \lint_Q g  + \epsilon \\
	 &\implies& \lint_Q f \le \lint_Q g
\end{eqnarray*}
\newpage


\begin{flushleft} 
\textbf{Class 18.101} - Section 10 Problem 3\\
\rule{500pt}{1pt}\\
\end{flushleft} 

Let $[0,1]^2 = [0,1] \times [0,1]$. Let $f : [0,1]^2 \to \mathbb{R}$ be defined by setting $f(x,y) = 0$ if $y \neq x$, and $f(x,y) = 1$ if $y = x$. Show that $f$ is integrable over $[0,1]^2$.\\

\noindent By the Riemann condition, for every $\epsilon$ we need to find a $P$ such that 
\[ U(f,P) - L(f,P) < \epsilon \]

First we note that for every rectangle $R$ that contains a point $(y,y)$ it is true that $\text{inf} \{ f(\textbf{x}) | \textbf{x} \in R\} = 0$. This is to say that every rectangle contains that contains a point $(y,y)$ contains a point $(x,y)$ such that $x \neq y$. This is clear from the definition of the rectangle $R = [x_i,x_{i+1}] \times  [y_i,y_{i+1}]$ which contains the points $(x_i,y_i),(x_{i+1},y_i),(x_i,y_{i+1}),(x_{i+1},y_{i+1})$ which cannot all be of the form $(y,y)$. \\
Because clearly $\text{inf} \{ f(\textbf{x}) | \textbf{x} \in R\} = 0$ for all $R$ which do not contain a point $(y,y)$ ($f(\textbf{x}) = 0$ on the whole rectangle) we know that $L(f,P) = 0$ for every partition.

This reduces the task down to proving that for every $\epsilon$ there exists a $P$ such that $U(f,P) < \epsilon$. 

\noindent For every $\epsilon$ there exists an $N$ such that $\frac{3}{N} < \epsilon$. Now let the partition be 
\[ P = \{ [0,\frac{1}{N},\frac{2}{N},\cdots,\frac{N-1}{N},1] , [0,\frac{1}{N},\frac{2}{N},\cdots,\frac{N-1}{N},1] \} \]

There will be $N$ rectangles along the diagonal that contain the points of the form $(y,y)$ and 2 diagonals next to the main diagonals (with $N-1$ points each) which have a corner of the form $(y,y)$. The remainder of the rectangles do not contain a point of the form $(y,y)$ so they will not contribute to the sum in $U(f,P)$.

\noindent Calculating $U(f,P)$ gives
\begin{eqnarray*} 
U(f,P) =   \sum_R M_R(f) \cdot \nu (R) &=& N(\frac{1}{N^2}) \cdot 1 + 2(N-1)\frac{1}{N^2} \cdot 1\\
&<& N(\frac{1}{N^2}) + 2(N)\frac{1}{N^2}  = \frac{3}{N} < \epsilon \\
\end{eqnarray*}
This completes the proof.

\begin{flushleft} 
\textbf{Class 18.101} - Section 10 Problem 4\\
\rule{500pt}{1pt}\\
\end{flushleft} 


We say $f : [0,1] \to \mathbb{R}$ is \emph{increasing} if $f(x_1) \le f(x_2)$ whenever $x_1 < x_2$. If $f,g : [0,1] \to \mathbb{R}$ are increasing and non-negative, show that the function $h(x,y) = f(x)g(y)$ is integrable over $[0,1]^2$.\\

The first thing we would like to show is $\text{sup} \: \{ h(x,y) \} = \text{sup} \: \{ f(x) \} \cdot \text{sup} \: \{ g(y) \}$ on any rectangle. The proof for the infimum is similar.

Since $0 < f(x)$ and $0 < g(y)$ we know that $0 < f(x) g(y)$. This means that if $F,G$ are least upper bounds for $f,g$ respectively then
\[ f(x) \le F, g(y) \le G \implies f(x) g(y) \le F G \quad \text{since $0 < F$ and $0 < G$} \]

\noindent So $FG$ is an upper bound of $h(x,y)$.\\
%If $H$ is the least upper bound of $h(x,y)$ then for a given $\epsilon$ there exists an $x',y'$ such that $H - \epsilon < h(x',y')$.
We now show that $FG$ is the \emph{least} upper bound.

For a given $0 < \epsilon$ it is the case that either $\epsilon < 1 + F + G$ or $1 + F + G \le \epsilon$. For the first case we let $\epsilon' = \frac{\epsilon}{1+ F + G}$ and in the second we let $\epsilon' =  \sqrt{\frac{\epsilon}{1+ F + G}}$

\noindent Since $F,G$ are least upper bounds of their respective functions their exists $x'$ and $y'$ such that 
\[ F - \epsilon' < f(x') \quad \text{and} \quad G - \epsilon' < g(y') \]

\noindent Performing the algebra of multiplying these inequalities we have 
\[ FG < \epsilon'^2 + \epsilon' f(x') + \epsilon' g(y') + f(x') g(y') \]

\noindent Now if $\epsilon < 1+F+G$ then $\epsilon' < 1$ so we have 
\begin{eqnarray*}
FG &<& \epsilon'^2 + \epsilon' f(x') + \epsilon' g(y') + f(x') g(y') \\
&<& \epsilon' + \epsilon' f(x') + \epsilon' g(y') + f(x') g(y') \\
&<& \epsilon' + \epsilon' F + \epsilon' G + f(x') g(y') \\
&<& \epsilon' (1 + F + G) + f(x') g(y') \\
&<& \epsilon + f(x') g(y') \\
\end{eqnarray*}

\noindent This shows that $FG - \epsilon < f(x') g(y')$. The proof for the case when $\epsilon \ge 1 + F + G$ is similar. We have show that for every $\epsilon$ there exists an $x',y'$ such that $FG - \epsilon < f(x') g(y')$ which says that $FG$ is the least upper bound.

We will show that $h(x,y)$ is integrable by satisfying the Riemann criterion. Let $H_m$ be $\inf \{ h(x,y) \}$ and  $H_M$ be $\sup \{ h(x,y) \}$. We let $N$ be such that $\frac{1}{N} < \frac{\epsilon}{2(H_M - H_m)}$ and let the partition be 
\[ P = \{ [0,\frac{1}{N},\frac{2}{N},\cdots,\frac{N-1}{N},1] , [0,\frac{1}{N},\frac{2}{N},\cdots,\frac{N-1}{N},1] \} \]

Then we have
\begin{eqnarray*}
\hspace{-15pt} U(h,P) - L(h,P) &=& \sum_R M(R) \nu(R) -  \sum_R m(R) \nu(R)\\
&=& \frac{1}{N^2} (\sum_R \sup \{ h(x,y) \} -  \sum_R \inf \{ h(x,y) \} ) \\
&=& \frac{1}{N^2} (\sum_R \sup \{ f(x) \} \cdot \sup \{ g(y) \} -  \sum_R \inf \{ f(x) \} \cdot \inf \{ g(y) \}  ) \\
&=& \frac{1}{N^2} (f(1)g(1) +  \sum_{i = 0}^{N-1} f(1)  g(\frac{i}{N}) + \sum_{i = 0}^{N-1} f(\frac{i}{N})  g(1) -  f(0)g(0) - \sum_{i = 0}^{N-1} f(0)  g(\frac{i}{N}) - \sum_{i = 0}^{N-1} f(\frac{i}{N})  g(0) ) \\
&<& \frac{1}{N^2} ((2N - 1) H_M -  (2N - 1) H_m ) =   \frac{2N -1}{N^2} (H_M - H_m) \\
&<&  \frac{2N}{N^2} (H_M - H_m)  = \frac{2}{N} (H_M - H_m) < \epsilon \\
\end{eqnarray*}

\newpage

\begin{flushleft} 
\textbf{Class 18.101} - Section 10 Problem 5\\
\rule{500pt}{1pt}\\
\end{flushleft} 

Let $f : \mathbb{R} \to \mathbb{R}$ be defined by setting $f(x) = \frac{1}{q}$ if $x = \frac{p}{q}$, where $p$ and $q$ are positive integers with no common factor, and $f(x) = 0$ otherwise. Show $f$ is integrable over $[0,1]$.


First we will show that a non-negative function $g : [a,b] \to \mathbb{R}$ with only a finite number of nonzero points can be partitioned such that $U(g,P) < \epsilon$ for any $\epsilon$.

Let $S = \{ x_1, x_2,\cdots,x_N \}$ be the location of the non-zero points and $M = \text{Max} \;\{ f(x_1),f(x_2), \cdots f(x_N) \}$. If we also let $d$ be the minimum distance among the points. For any given $\epsilon$ we can let $d' = \text{Min} \: \{ \frac{d}{3}, \frac{\epsilon}{M N} \} $. We take $d'$ to be less than equal to $\frac{d}{3}$ so that when we create the partition there is no overlap between subintervals. Take the partition to contain $x_i \in S$ and $x_i \pm d'$ depending on which sign keeps the point within the domain. The rest of the points chosen outside the subintervals defined by $[x_i,x_i \pm d']$ can be chosen arbitrarily since y will not contribute to the sum. Calculating $U(g,P)$ gives

\begin{eqnarray*}
U(g,P) &=& \sum_R M_R( \{ g(R) \}) \nu(R) = d' \sum f(x_i) \\
&=& d' \sum M = d' N M \le \frac{\epsilon}{MN} MN = \epsilon \\
\end{eqnarray*}

This shows that the upper sum of a function which has a finite number of non zero points that can be made arbitrarily small.

Next we show that the given function $f$ can be bounded by a $g + \frac{\epsilon}{2}$ where $g$ is a finite function and $\epsilon$ is a positive quantity.
	Let $\frac{\epsilon}{2} > 0$ and we ask how many points of $f$ are greater than this value. If we let $q$ be the largest integer such that $\frac{\epsilon}{2} < \frac{1}{q}$ then there are at most $q-1$ values for $p$ such that $f(\frac{p}{q}) = \frac{1}{q}$ since $0 \le \frac{p}{q} \le 1$. Similarly for $\frac{\epsilon}{2} < \frac{1}{q-1}$ there must be $q-2$ points such that $f(\frac{p}{q-1}) = \frac{1}{q-1}$. Continuing in this way there will be a total of $(q-1) + (q-2) + \cdots + 2 + 1 = \frac{q(q+1)}{2} - q$ points $f(x) > \frac{\epsilon}{2}$. This this number is finite we have shown that there are a finite number of points $\{ x_1,x_2,\cdots, x_n \}$ such that $f(x_i) > \frac{\epsilon}{2}$.

Let $g$ be a function such that $g(x) = 1$ for $x \in \{ x_1,x_2,\cdots,x_n \}$ and $g(x) = 0$ otherwise. It is clear then that $f(x) < g(x) + \frac{\epsilon}{2}$ for all $x \in [0,1]$. By construction $f(x) < \frac{\epsilon}{2} = \frac{\epsilon}{2} + 0 =  \frac{\epsilon}{2} + g(x)$ for $x \notin \{ x_1,x_2,\cdots, x_n \}$ and $f(x) = g(x) < g(x) + \frac{\epsilon}{2}$ for $x \in \{ x_1,x_2,\cdots, x_n \}$. So it is the case that
\[ f(x) < g(x) + \frac{\epsilon}{2} \quad \text{for $x \in [0,1]$} \]
Since $U(c,P) = c \cdot \nu (P)$ when $c$ is a constant we have $U(\epsilon,P) = \epsilon \cdot \nu(P) = \epsilon \cdot 1 = \epsilon$ since the partition is of the interval $[0,1]$. By the result about finite functions we may pick a partition of $g$ such that
\[ U(g,P) < \frac{\epsilon}{2} \]

We may now calculate as follows
\begin{eqnarray*}
U(f,P) < U(g,P) &=& \sum_R M(R) \nu(R) \\ 
&=& \sum_R \sup ( \{ g(x) \} + \frac{\epsilon}{2} ) \nu(R) = \sum_R ((\sup \{ g(x)\})  + \frac{\epsilon}{2} ) \nu(R) \\ 
&=& \sum_R \sup ( \{ g(x) \} \nu(R) + \sum_R \frac{\epsilon}{2} \:\nu(R) =  \sum_R \sup ( \{ g(x) \} \nu(R) + \frac{\epsilon}{2} \\ 
&=& U(g,P) + \frac{\epsilon}{2} < \frac{\epsilon}{2} + \frac{\epsilon}{2} = \epsilon \\
\end{eqnarray*}


\end{document}
