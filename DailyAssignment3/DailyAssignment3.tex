%\documentclass[11pt,reqno]{amsart}
\documentclass[11pt,reqno]{article}
\usepackage[margin=.8in, paperwidth=8.5in, paperheight=11in]{geometry}
%\usepackage{geometry}                % See geometry.pdf to learn the layout options. There are lots.
%\geometry{letterpaper}                   % ... or a4paper or a5paper or ... 
%\geometry{landscape}                % Activate for for rotated page geometry
%\usepackage[parfill]{parskip}    % Activate to begin paragraphs with an empty line rather than an indent7
\usepackage{graphicx}
\usepackage{pstricks}
\usepackage{amssymb}
\usepackage{epstopdf}
\usepackage{amsmath}
\usepackage{subfigure}
\usepackage{caption}
\pagestyle{plain}
%\renewcommand{\topfraction}{0.3}
%\renewcommand{\bottomfraction}{0.8}
%\renewcommand{\textfraction}{0.07}
\DeclareGraphicsRule{.tif}{png}{.png}{`convert #1 `dirname #1`/`basename #1 .tif`.png}

\title{Real Analysis $\mathbb{II}$: \\ Daily Assignment 3}
\author{Andrew Rickert}
\date{Started: February 27, 2012 \\ \hspace{1pt} Ended: February ??,  2012}                                           % Activate to display a given date or no date

\begin{document}
\maketitle


% Page 1
\begin{flushleft} 
\textbf{Class 18.101} - Section 5 Problem 2\\
\rule{500pt}{1pt}\\
\end{flushleft} 

We have the function defined by

\begin{eqnarray*} 
f(\textbf{0}) &=& 0 \\
f(x,y) &=& \frac{x y}{x^2 + y^2} \quad \text{if $(x,y) \neq 0$} \\
\end{eqnarray*}

\noindent (a) For which vectors $\textbf{u} \neq 0$ does $f'(\textbf{0};\textbf{u})$ exist? Evaluate it when it exists.\\

Let $\textbf{u} = \left[ \begin{array}{c} h\\ k\\ \end{array} \right]$ and substitute into the expression for the directional derivative:
\[ \phi(t) = \frac{f(\textbf{0} + t \textbf{u}) - f(\textbf{0})}{t} = \frac{1}{t} \frac{t^2 h k}{t^2 (h^2 + k^2)} = \frac{h k}{t (h^2 + k^2)} \]
If both $h \neq 0$ and $k \neq 0$ then the quotient does not have a finite limit. If we have $h = 0$ and $k \neq 0$ or $h \neq 0$ and $k = 0$ then $q(t) = 0$ so $\lim_{t \to 0} q(t) = f'(\textbf{0};\textbf{u}) = 0$.\\

\noindent (b) Do $D_1 f$ and $D_2 f$ exist at $\textbf{0}$?\\

In part $a$ it was shown that the directional derivative exists for $\textbf{u} = \left[ \begin{array}{c} h\\ k\\ \end{array} \right]$ when $h = 0$ or $k = 0$ but not both. Since $D_1 f(\textbf{0}) = f'(\textbf{0};\textbf{e}_1)$, if we let $h = 1$ and $k = 0$ then we satisfy the conditions for the directional derivative to exist at $\textbf{e}_1$ so $D_1 f$ exists. Letting $h = 0$ and $k = 1$ and using the same reasoning we get find that $D_2 f$ exists.\\

\noindent (c ) Is $f$ differentiable at $\textbf{0}?$\\

If any of the directional derivatives do not exist then $f$ is not differentiable. Suppose the directional derivative at $\textbf{a}$ in the $\textbf{u}$ direction does not exist. If $f$ were differentiable then we have 
\[ f'(\textbf{a};\textbf{u}) = Df(\textbf{a}) \cdot \textbf{u} \]
which implies the existence of the directional derivative at the value and direction it does not exist. This is a contradiction. \\
\indent Since the directional derivative does not exist when $h \neq 0$ and $k \neq 0$ at $\textbf{0}$ it follows that $f$ is not differentiable at $\textbf{0}$. \\

\noindent (d) Is $f$ continuous at $\textbf{0}?$

\vspace{15pt}
\begin{flushleft} 
\textbf{Class 18.101} - Section 5 Problem 3\\
\rule{500pt}{1pt}\\
\end{flushleft} 



\vspace{15pt}
\begin{flushleft} 
\textbf{Class 18.101} - Section 5 Problem 4\\
\rule{500pt}{1pt}\\
\end{flushleft} 

 
 
\vspace{15pt}
\begin{flushleft} 
\textbf{Class 18.101} - Section 5 Problem 5\\
\rule{500pt}{1pt}\\
\end{flushleft} 



\vspace{15pt}
\begin{flushleft} 
\textbf{Class 18.101} - Section 5 Problem 7\\
\rule{500pt}{1pt}\\
\end{flushleft} 



\end{document}
