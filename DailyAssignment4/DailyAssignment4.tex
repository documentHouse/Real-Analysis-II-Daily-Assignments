%\documentclass[11pt,reqno]{amsart}
\documentclass[11pt,reqno]{article}
\usepackage[margin=.8in, paperwidth=8.5in, paperheight=11in]{geometry}
%\usepackage{geometry}                % See geometry.pdf to learn the layout options. There are lots.
%\geometry{letterpaper}                   % ... or a4paper or a5paper or ... 
%\geometry{landscape}                % Activate for for rotated page geometry
%\usepackage[parfill]{parskip}    % Activate to begin paragraphs with an empty line rather than an indent7
\usepackage{graphicx}
\usepackage{pstricks}
\usepackage{amssymb}
\usepackage{epstopdf}
\usepackage{amsmath}
\usepackage{subfigure}
\usepackage{caption}
\pagestyle{plain}
%\renewcommand{\topfraction}{0.3}
%\renewcommand{\bottomfraction}{0.8}
%\renewcommand{\textfraction}{0.07}
\DeclareGraphicsRule{.tif}{png}{.png}{`convert #1 `dirname #1`/`basename #1 .tif`.png}

\title{Real Analysis $\mathbb{II}$: \\ Daily Assignment 4}
\author{Andrew Rickert}
\date{Started: March 1, 2012 \\ \hspace{1pt} Ended: March 2,  2012}                                           % Activate to display a given date or no date

\begin{document}
\maketitle

% Page 1
\begin{flushleft} 
\textbf{Class 18.101} - Section 6 Problem 2\\
\rule{500pt}{1pt}\\
\end{flushleft} 

\noindent Define $f: \mathbb{R} \to \mathbb{R}$ by setting $f(0) = 0$ and
\[ f(t) = t^2 \sin(1/t) \quad \text{if} \quad t \neq 0 \]

\noindent (a) Show $f$ is differentiable at $0$, and calculate $f'(0)$\\

We show that the derivative exists by calculating it at $t = 0$. By the definition of the derivative
\begin{eqnarray*}
\lim_{\textbf{h} \to \textbf{0}} \frac{f(\textbf{0} + \textbf{h}) - f(\textbf{0}) - \textbf{B}\cdot \textbf{h}}{|\textbf{h}|} &=&  \lim_{t \to 0} \frac{f(0 + t) - f(0) - B \cdot t}{|t|} \quad \text{implies right and left hand derivative exist}\\
&=& \lim_{t \to 0} \frac{f(t)}{|t|} \quad \text{we are showing $B$ = 0 to be the derivative}\\
&=& \lim_{t \to 0} \frac{t^2}{|t|} \sin(\frac{1}{t})\\
\end{eqnarray*}

\noindent Since it is the case that $-1 \le \sin(\frac{1}{t}) \le 1$ we have the following
\begin{eqnarray*}
-|t| \le \frac{t^2}{|t|} \sin(\frac{1}{t}) \le |t| &\implies& \lim_{t \to 0} -|t| \le \lim_{t \to 0} \frac{t^2}{|t|} \sin(\frac{1}{t}) \le \lim_{t \to 0} |t| \\
& \implies & 0 \le \lim_{t \to 0} \frac{t^2}{|t|} \sin(\frac{1}{t}) \le 0 \\
& \implies & \lim_{t \to 0} \frac{t^2}{|t|} \sin(\frac{1}{t}) = 0
\end{eqnarray*}

\noindent This shows that $f$ is differentiable at 0 and that $f'(0) = 0$.\\

\noindent (b) Calculate $f'(t)$ if $t \neq 0$.\\

Since $\frac{1}{t}$ is differentiable at $t \neq 0$ and $t^2, \sin t$ are differentiable everywhere we can conclude by they chain rule that $t^2 \sin (\frac{1}{t} )$ is differentiable everywhere except $t = 0$. The chain rule gives the formula
\[ f'(t) = 2t \sin( \frac{1}{t} ) - \cos(\frac{1}{t}) \quad \text{for $t \neq 0$}\]

\noindent (c) Show $f'$ is not continuous at 0.\\

$f'(t)$ is continuous at 0 if and only if $\lim_{t \to 0} f'(t) = f'(0)$. The first term in is bounded as 
\[ -2|t| \le 2t \sin( \frac{1}{t} ) \le 2|t| \]
so the $\lim_{t \to 0} 2t \sin( \frac{1}{t} ) = 0$. However the limit $\lim_{t \to 0} \cos (\frac{1}{t})$ clearly does not exist since 
\[ \lim_{n \to \infty} t(n) = \frac{1}{2 \pi n} = 0 \quad \text{but} \quad \cos(\frac{1}{t(n)}) = \cos 2 \pi n = 1 \]

\noindent this shows that $f'$ can not be continuous as 0.\\

\noindent (d) Conclude that $f$ is differentiable on $\mathbb{R}$ but not of class $C^1$ on $\mathbb{R}$.\\

In part $a$ it was shown that $f$ is differentiable at $t = 0$ and in part $b$ it was shown that $f$ is differentiable for all other values so $f$ is differentiable on $\mathbb{R}$. Since part $c$ showed that $f'$ is not continuous at $t = 0$ we know that $f$ is not continuously differentiable on any interval that contains $t = 0$. This says that for $t \in \mathbb{R}$ that $f \notin C^1$.

\begin{flushleft} 
\textbf{Class 18.101} - Section 6 Problem 5\\
\rule{500pt}{1pt}\\
\end{flushleft} 

\noindent Let $f: \mathbb{R}^2 \to \mathbb{R}^2$ be defined by the equation
\[ f(r,\theta) = (r \cos \theta, r \sin \theta) \]

\noindent (a) Calculate $D f$ and det $D f$.

\noindent If we let $f_1(r,\theta) = r \cos \theta$ and $f_2(r,\theta) = r \sin \theta$ then jacobian matrix is of the form
\[ 
\left( \begin{array}{cc}
D_1 f_1 & D_2 f_1 \\
D_1 f_2 & D_2 f_2  \\
\end{array} \right)
\implies 
\left( \begin{array}{cc}
\cos \theta & -r \sin \theta \\
\sin \theta & r \cos \theta  \\
\end{array} \right)
\]

\noindent Taking determinant
\[ \text{det} \; D f = \cos \theta \cdot r \cos \theta - ( - r \sin \theta \cdot \sin \theta ) = r ( \cos^2 \theta + \sin^2 \theta) = r \]

\noindent (b) Sketch the image under $f$ of the set $S = [1,2] \times [0,\pi]$. \\ 


\begin{flushleft} 
\textbf{Class 18.101} - Section 6 Problem 9\\
\rule{500pt}{1pt}\\
\end{flushleft} 

\noindent Let $g: \mathbb{R} \to \mathbb{R}$ be a function of class $C^2$. Show that 
\[ \lim_{h \to 0} \frac{g(a + h) - 2 g(a) + g(a - h)}{h^2}  = g''(a)\]

\noindent Let $f(x) = g(x) - g(x - h)$. We may now recast the problem:
\[ \phi(h) = \frac{g(a + h) - 2 g(a) + g(a - h)}{h^2} = \frac{f(a+h) - f(a)}{h^2} \]

\noindent Since $g \in C^2$ we can apply the mean value theorem to the above expression and get
\[ \frac{f(a+h) - f(a)}{h^2} = \frac{f'(h_0) h}{h^2} = \frac{f'(h_0)}{h} \quad \text{for $a < h_0 < a + h$} \]

This expression can be rewritten as $f'(h_0) = g'(h_0) - g'(h_0 - h)$. Because $g' \in C^1$ we may apply the mean value theorem again to the above expression to get 
\[  \frac{f'(h_0)}{h} =  \frac{g'(h_0) - g'(h_0 - h)}{h} = \frac{g''(h_1) h}{h} = g''(h_1) \quad \text{for $h_0 - h < h_1 < h_0$ } \]

To finish we need to evaluate $\lim_{h \to 0} g''(h_1)$. Since we have
\begin{eqnarray*}
& & a < h_0 < a + h \quad \text{and} \quad h_0 - h < h_1 < h_0 \\
& \implies & a + h < h_0 + h < a + 2 h \quad \text{and} \quad h_0 < h_1 + h < h_0 + h \\
& \implies & a < h_1 + h < a + 2 h \\
& \implies & a - h < h_1 < a + h 
\end{eqnarray*}
The final expression is true for all $h$ so we have
\begin{eqnarray*}
\lim_{h \to 0} a - h & < & \lim_{h \to 0} h_1 < \lim_{h \to 0} a + h \implies a \le \lim_{h \to 0} h_1 \le  a\\
\implies \lim_{h \to 0} h_1 = a
\end{eqnarray*}

\noindent Because $g''$ is continuous we may carry out the limit to prove the theorem

\[ \lim_{h \to 0} \phi(t) = \lim_{h \to 0} g''(h_1) = g''(a) \]

\begin{flushleft} 
\textbf{Class 18.101} - Section 6 Problem 10\\
\rule{500pt}{1pt}\\
\end{flushleft} 

\noindent Define $f : \mathbb{R}^2 \to \mathbb{R}$ by setting $f(\textbf{0}) = 0$ and 
\[ f(x,y) = \frac{xy(x^2 - y^2)}{x^2 + y^2} \quad \text{if} \quad \text{$(x,y) \neq \textbf{0}$} \]

\noindent (a) Show $D_1 f$ and $D_2 f$ exist at $\textbf{0}$.\\

\noindent Start with $D_1 f$ which is defined as the directional derivative in the $\textbf{e}_1$ direction:

\begin{eqnarray*}
D_1 f &=& f'(\textbf{0};\textbf{e}_1) = \lim_{t \to 0} \frac{f(\textbf{0} + t \textbf{e}_1) - f(\textbf{0})}{t}\\
&=& \lim_{t \to 0} \frac{f(t \textbf{e}_1)}{t} = \lim_{t \to 0} 0 = 0\\
\end{eqnarray*}
Similarly
\begin{eqnarray*}
D_2 f &=& f'(\textbf{0};\textbf{e}_1) = \lim_{t \to 0} \frac{f(\textbf{0} + t \textbf{e}_2) - f(\textbf{0})}{t}\\
&=& \lim_{t \to 0} \frac{f(t \textbf{e}_2)}{t} = \lim_{t \to 0} 0 = 0\\
\end{eqnarray*}

\noindent (b) Calculate $D_1 f$ and $D_2 f$ at $(x,y) \neq 0$. \\

The differentiability of $f(x,y)$ when $(x,y) \neq 0$ follows from the differentiability of sums, differences, products, quotients, and compositions of polynomials. This means we can find the partial derivatives using the standard rules of calculus:
\begin{eqnarray*}
D_1 f (x,y) = y \left( \frac{x^4 + 4 x^2 y^2 - y^4}{(x^2 + y^2)^2} \right)\\
D_2 f (x,y) = x \left( \frac{x^4 - 4 x^2 y^2 - y^4}{(x^2 + y^2)^2} \right)\\
\end{eqnarray*}

\noindent (c) Show $f$ is of class $C^1$ on $\mathbb{R}^2$.\\

By the continuity of sums, differences, products, quotients, and compositions of polynomials the expressions for $D_1 f$ and $D_2 f$ in part $b$ are continuous for $(x^2 + y^2)^2 \neq 0$. This is true for all $x,y$ except $x = y = 0$. Lastly we need to show that $D_1 f$ and $D_2 f$ are continuous at $\textbf{0}$. This amounts to showing that $\lim_{ \textbf{h} \to \textbf{0}} D_1 f(\textbf{h}) = D_1 f(\textbf{0}) = 0$ and  $\lim_{ \textbf{h} \to \textbf{0}} D_2 f(\textbf{h}) = D_2 f(\textbf{0}) = 0$.\\
\noindent We handle $D_1 f$ first. Let 
\[ g_1(x,y) =  \frac{x^4 + 4 x^2 y^2 - y^4}{(x^2 + y^2)^2} = \frac{(x^2-y^2)(x^2 + y^2) + 4 x^2 y^2}{(x^2 + y^2)^2} =  \frac{x^2-y^2}{x^2 + y^2}  + \frac{4 x^2 y^2}{(x^2 + y^2)^2} \]

\noindent We can now derive some bounds on $g(x,y)$ that will be useful
\begin{eqnarray*}
g_1(x,y) > \frac{-x^2-y^2}{x^2 + y^2}  + \frac{4 x^2 y^2}{(x^2 + y^2)^2} = -1 + \frac{4 x^2 y^2}{(x^2 + y^2)^2} >  -1\\
\end{eqnarray*}
\noindent Also
\begin{eqnarray*}
g_1(x,y) &<& \frac{x^2 + y^2}{x^2 + y^2}  + \frac{4 x^2 y^2}{(x^2 + y^2)^2} = 1 + \frac{4 x^2 y^2}{(x^2 + y^2)^2}\\
&=& 1 + \frac{4 x^2 y^2}{x^4 + y^4 + 2 x^2 y^2} < 1 + \frac{4 x^2 y^2}{2 x^2 y^2} = 3 \\
\end{eqnarray*}

\noindent This gives us the bounds
\[ -1 < g_1(x,y) < 3 \]

\noindent For $D_2 f$ we may perform similar algebra for $g_2(x,y)$ defined as follows:
\[ g_2(x,y) =  \frac{x^4 - 4 x^2 y^2 - y^4}{(x^2 + y^2)^2} = \frac{(x^2-y^2)(x^2 + y^2) - 4 x^2 y^2}{(x^2 + y^2)^2} =  \frac{x^2-y^2}{x^2 + y^2}  - \frac{4 x^2 y^2}{(x^2 + y^2)^2} \]

\noindent For the lower bound
\begin{eqnarray*}
g_1(x,y) &>& \frac{-x^2 - y^2}{x^2 + y^2}  - \frac{4 x^2 y^2}{(x^2 + y^2)^2} = -1 - \frac{4 x^2 y^2}{(x^2 + y^2)^2}\\
&=& 1 - \frac{4 x^2 y^2}{x^4 + y^4 + 2 x^2 y^2} > -1 - \frac{4 x^2 y^2}{2 x^2 y^2} = -3 \\
\end{eqnarray*}
\noindent As well for the upper bound
\begin{eqnarray*}
g_1(x,y) < \frac{x^2+y^2}{x^2 + y^2}  - \frac{4 x^2 y^2}{(x^2 + y^2)^2} = 1 - \frac{4 x^2 y^2}{(x^2 + y^2)^2} < 1\\
\end{eqnarray*}
\noindent So the bounds for $g_2(x,y)$ are
\[ -3 < g_2(x,y) < 1 \] 

This allows us to say that $-3 < g_1(x,y), g_2(x,y) < 3$ for all $(x,y)$. We now show that\\ $\lim_{\textbf{h} \to 0} D_1 f(\textbf{h}) = 0$ with the proof for $D_2 f$ being analogous.\\

\noindent From the definition of $g_1(x,y)$ and it's bounds we have 
\begin{eqnarray*}
 & &-3 y < y \;g_2(x,y) = D_1 f(x,y) < 3 y \quad \text{now let $\textbf{h} = (h_1,h_2)$ so that} \\
 &\implies& 0 = \lim_{\textbf{h} \to 0} - 3 h_2 \le  \lim_{\textbf{h} \to 0} D_1 f(\textbf{h}) \le \lim_{\textbf{h} \to 0}  3 h_2 = 0 \\
 &\implies&  \lim_{\textbf{h} \to 0} D_1 f(\textbf{h}) = 0
\end{eqnarray*}


\noindent (d) Show that $D_2 D_1 f$ and $D_1 D_2 f$ exist at $\textbf{0}$, but are not equal there.\\

We use the definition of the directional derivative to show the second partials exist. For $D_2 D_1 f(\textbf{0})$ we have

\begin{eqnarray*}
D_2 D_1 f(\textbf{0}) &=& \lim_{t \to 0} \frac{D_1 f(\textbf{0} + t \textbf{e}_2) - D_1 f(\textbf{0})}{t} =  \lim_{t \to 0} \frac{D_1 f(t \textbf{e}_2)}{t}\\
			&=& \lim_{t \to 0} \frac{1}{t} (-t) = \lim_{t \to 0} -1  = -1 \\
\end{eqnarray*}

\noindent For $D_1 D_2 f(\textbf{0})$ we have

\begin{eqnarray*}
D_1 D_2 f(\textbf{0}) &=& \lim_{t \to 0} \frac{D_2 f(\textbf{0} + t \textbf{e}_1) - D_2 f(\textbf{0})}{t} =  \lim_{t \to 0} \frac{D_2 f(t \textbf{e}_1)}{t}\\
			&=& \lim_{t \to 0} \frac{1}{t} (t) = \lim_{t \to 0} 1  =  1\\
\end{eqnarray*}

\noindent So we have $D_1 D_2 f(\textbf{0}) \neq D_2 D_1 f(\textbf{0})$.

\end{document}
