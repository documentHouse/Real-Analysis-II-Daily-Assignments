%\documentclass[11pt,reqno]{amsart}
\documentclass[11pt,reqno]{article}
\usepackage[margin=.8in, paperwidth=8.5in, paperheight=11in]{geometry}
%\usepackage{geometry}                % See geometry.pdf to learn the layout options. There are lots.
%\geometry{letterpaper}                   % ... or a4paper or a5paper or ... 
%\geometry{landscape}                % Activate for for rotated page geometry
%\usepackage[parfill]{parskip}    % Activate to begin paragraphs with an empty line rather than an indent7
\usepackage{graphicx}
\usepackage{pstricks}
\usepackage{amssymb}
\usepackage{epstopdf}
\usepackage{amsmath}
\usepackage{subfigure}
\usepackage{caption}
\pagestyle{plain}

\def\uint{\mathchoice%
    {\mkern13mu\overline{\vphantom{\intop}\mkern9mu}\mkern-20mu}%
    {\mkern9mu\overline{\vphantom{\intop}\mkern9mu}\mkern-14mu}%
    {\mkern9mu\overline{\vphantom{\intop}\mkern9mu}\mkern-14mu}%
    {\mkern9mu\overline{\vphantom{\intop}\mkern9mu}\mkern-14mu}%
  \int}
\def\lint{\mkern3mu\underline{\vphantom{\intop}\mkern9mu}\mkern-10mu\int}

\DeclareGraphicsRule{.tif}{png}{.png}{`convert #1 `dirname #1`/`basename #1 .tif`.png}

\title{Real Analysis $\mathbb{II}$: \\ Daily Assignment 9}
\author{Andrew Rickert}
\date{Started: March 29, 2012 \\ \hspace{1pt} Ended: April 1,  2012}                                           % Activate to display a given date or no date

\begin{document}
\maketitle


% Page 1
\begin{flushleft} 
\textbf{Class 18.101} - Section 12 Problem 1\\
\rule{500pt}{1pt}\\
\end{flushleft} 

\noindent Carry out Step 2 of the proof of Theorem 12.2.\\

The proof is the same as that for the lower integral mutatis mutandis.

\begin{flushleft} 
\textbf{Class 18.101} - Section 12 Problem 2\\
\rule{500pt}{1pt}\\
\end{flushleft} 

Let $I = [0,1]$; let $Q = I \times I$. Define $f : Q \to \mathbb{R}$ by letting $f(x,y) = \frac{1}{q}$ if $y$ is rational and $x = \frac{p}{q}$, where $p$ and $q$ are positive integers with no common factor; let $f(x,y) = 0$ otherwise.\\

\noindent (a) Show that $\int_Q f$ exists.\\

$f(x,y)$ vanishes are everywhere except at rational $x$ and $y$. This means the only discontinuities may occur at the rational points of $[0,1]$. For $x$ there are a countable number $y$ such that $f(x,y) \neq 0$. Since there a countable number of $x$ there is at most a countable set of countable sets of discontinuities. A countable collection of countable sets is countable so we may apply theorem 11.2 that says if $f$ is discontinuous only on a set of measure zero then it is integrable. We have shown the number of possible discontinuities is at most countable so the theorem applies.\\

\noindent (b) Compute
\[ \lint_{y \in I} f(x,y) \quad \text{and} \quad \uint_{y \in I} f(x,y)\]

For the lower integral we note that whether $x_0$ is rational or not there will be irrational values in subinteval of $x_0 \times [0,1]$. This means that for all $x$ we have $L(f,P) = 0$ for any partition so we get 
\[ \lint_{y \in I} f(x,y)  = 0 \]

For the upper integral we need to distinguish whether $x$ is rational or not. If we suppose that it is irrational then $f(x,y) = 0$ for all $(x,y) \in \{x\} \times [0,1]$. It is clear in this case that 
\[  \uint_{y \in I} f(x,y)  = 0 \quad \text{for irrational $x$} \]
If we fix $x = \frac{p}{q}$ then $f(x,y) = \frac{1}{q}$ for rational $y$. Since the rationals are dense for every subinterval of $[0,1]$ we find for every partition $P$ we get $U(f,P) = \frac{1}{q}$. This implies
\[  \uint_{y \in I} f(x,y)  = \frac{1}{q} \quad \text{for rational $x$} \]

\noindent (c) Verify Fubini's theorem.\\

\noindent The lower integral $\lint_{y \in Q} f = 0$ for all $x$ so it is clear that 
\[ \int_{x \in Q} \lint_{y \in Q} f(x,y) = 0 \]

We define a function $I(x)$ to be the value of the upper integral for each fixed $x$. With this definition we have for $x \in [0,1]$, $I(x) = \frac{1}{q}$ for rational $x = \frac{p}{q}$ and $I(x) = 0$ otherwise. By theorem 11.3 a if $f$ vanishes except on a set of measure zero, then $\int I = 0$. The set of rationals in $[0,1]$ is of measure zero so we have 
\[ \int_{x \in Q} \uint_{y \in Q} f(x,y) =  \int I(x) = 0 \]

\noindent So we have 
\[  \int_{x \in Q} \lint_{y \in Q} f(x,y) = \int_{x \in Q} \uint_{y \in Q} f(x,y) = 0 \]
Fubini's theorem is verified.\\

\begin{flushleft} 
\textbf{Class 18.101} - Section 12 Problem 3\\
\rule{500pt}{1pt}\\
\end{flushleft} 

Let $Q = A \times B$, where $A$ is a rectangle in $\mathbb{R}^k$ and $B$ is a rectangle in $\mathbb{R}^n$. Let $f : Q \to \mathbb{R}$ be a bounded function.\\

\noindent (a) Let $g$ be a function such that
\[ \lint_{\textbf{y} \in B} f(\textbf{x},\textbf{y}) \le g(\textbf{x}) \le \uint_{\textbf{y} \in B} f(\textbf{x},\textbf{y}) \]
for all $\textbf{x} \in A$. Show that if $f$ is integrable over $Q$, then $g$ is integrable over $A$, and $\int_Q f = \int_A g$.\\

By Fubini's theorem which, also says the functions $\underline{I}(\textbf{x}) = \lint_{\textbf{y} \in B} f(\textbf{x},\textbf{y})$ and $\overline{I}(\textbf{x}) = \uint_{\textbf{y} \in B} f(\textbf{x},\textbf{y})$ are integrable, we have the following
\begin{eqnarray*}
\int_Q f = \int_{\textbf{x} \in A} \lint_{\textbf{y} \in B} f(\textbf{x},\textbf{y}) &=&  \lint_{\textbf{x} \in A} \lint_{\textbf{y} \in B} f(\textbf{x},\textbf{y}) \le \lint_{\textbf{x} \in A} g(\textbf{x}) \\
\lint_{\textbf{x} \in A} g(\textbf{x}) &\le& \uint_{\textbf{x} \in A} g(\textbf{x}) \\
\uint_{\textbf{x} \in A} g(\textbf{x}) = \uint_{\textbf{x} \in A} \uint_{\textbf{y} \in B} f(\textbf{x},\textbf{y}) &=& \int_{\textbf{x} \in A} \uint_{\textbf{y} \in B} f(\textbf{x},\textbf{y}) = \int_Q f 
\end{eqnarray*}
%\[ \hspace{-50pt} \int_Q f = \int_{\textbf{x} \in A} \lint_{\textbf{y} \in B} f(\textbf{x},\textbf{y}) =  \lint_{\textbf{x} \in A} \lint_{\textbf{y} \in B} f(\textbf{x},\textbf{y}) \le \lint_{\textbf{x} \in A} g(\textbf{x}) \le  \uint_{\textbf{x} \in A} g(\textbf{x}) = \uint_{\textbf{x} \in A} \uint_{\textbf{y} \in B} f(\textbf{x},\textbf{y}) = \int_{\textbf{x} \in A} \uint_{\textbf{y} \in B} f(\textbf{x},\textbf{y}) = \int_Q f  \]
Using these three expressions we derive
\[ \int_Q f \le  \lint_{\textbf{x} \in A} g(\textbf{x}) \le \uint_{\textbf{x} \in A} g(\textbf{x}) \le \int_Q f\]

This shows that the upper and lower integrals of $g$ are the same which shows that $g$ is integrable.  The common value of the upper and lower integrals of $g$ which this value of the integral must equal the bounds in the above expression so we have 
\[  \int_Q f = \int_{\textbf{x} \in A} g(\textbf{x}) \]
 
\noindent (b) Give an example where $\int_Q f$ exists and one of the iterated integrals
\[  \int_{\textbf{x} \in A} \int_{\textbf{y} \in B} f(\textbf{x},\textbf{y}) \quad \text{and} \quad  \int_{\textbf{y} \in B} \int_{\textbf{x} \in A} f(\textbf{x},\textbf{y})  \]
exist, but the other does not.\\
 
Let $f : [0,1]^2 \to [0,1]$ so that if $x = \frac{1}{2}$ then $f(x,y) = 1$ if $y$ is irrational and $f(x,y) = 0$ if $y$ is rational. We define $f$ to vanish everywhere else.

Because the rectangle $[\frac{1}{2},\frac{1}{2} + \epsilon] \times [0,1]$ has volume $\epsilon$ and covers the only nonzero portion of the function we've shown that the set of discontinuities of $f$ has measure 0. This means by theorem 11.2 that $\int_Q f$ exists where $Q =  [0,1]^2$.
For each $y$ the following integration exists since for each fixed $y$ there is only one non-zero point of $f$:
\[ \int_{x \in [0,1]} f(x,y) = 0\]
The remaining function $I(y) =  \int_{x \in [0,1]} f(x,y) = 0$ for all $y \in [0,1]$ is clearly integrable so 
\[ \int_{y \in [0,1]} \int_{x \in [0,1]} f(x,y)  \]
exists.
The other integral does not exist since the following integral does not exist
\[ \int_{y \in [0,1]} f(\frac{1}{2},y)  \]

\noindent (c) Find an example where both the iterated integrals of (b) exist, but the integral $\int_Q f$ does not.

\begin{flushleft} 
\textbf{Class 18.101} - Section 12 Problem 4\\
\rule{500pt}{1pt}\\
\end{flushleft} 

\noindent Let $A$ be open in $\mathbb{R}^2$; let $f : A \to \mathbb{R}$ be of class $C^2$. Let $Q$ be a rectangle contained in $A$.

\noindent (a) Use Fubini's theorem and the fundamental theorem of calculus to show that
\[ \int_Q D_2 D_1 f = \int_Q D_1 D_2 f \]

We know that $D_2 D_1 f$ is continuous since $C^2$ so we may rewrite this as 
\[ D_2 g \quad \text{where} \; g = D_1 f \]
So $g$ must be continuous since again $f$ is $C^2$.
Let $Q = [x_0,x_1] \times [y_0,y_1]$ so by Fubini's we may write 
\[ \int_Q D_2 g = \int_{x \in [x_0,x_1]} \int_{y \in [y_0,y_1]} D_2 g \]
Note the integral $ \int_{y \in [y_0,y_1]} $ is not the upper or lower integral because $D_2 D_1 f$ is continuous.\\
Since $g$ is continuous we may apply the fundamental theorem of calculus to get
\begin{eqnarray*} 
\int_{x \in [x_0,x_1]} \int_{y \in [y_0,y_1]} D_2 g(x,y) &=&  \int_{x \in [x_0,x_1]} g(x,y_1) - g(x,y_0) \\
&=& \int_{x \in [x_0,x_1]} D_1 f(x,y_1) - D_1 f(x,y_0)\\
&=& \int_{x \in [x_0,x_1]} D_1(f(x,y_1) - f(x,y_0)) \\
\end{eqnarray*}

Since $D_1(f(x,y_1) - f(x,y_0))$ is continuous we may use the fundamental theorem of calculus again to get

\begin{eqnarray*}  
\int_{x \in [x_0,x_1]} D_1(f(x,y_1) - f(x,y_0)) &=& (f(x_1,y_1) - f(x_1,y_0)) - (f(x_0,y_1) - f(x_0,y_0))
\end{eqnarray*}

Fubini's theorem applies to both orders of integration, the proof is the same, so may perform a similar calculation with $D_1 D_2 f$. Let $g = D_2 f$ so we derive.
 
\begin{eqnarray*}
\int_Q D_1 g &=& \int_{y \in [y_0,y_1]} \int_{x \in [x_0,x_1]} D_1 g(x,y) \\
&=&  \int_{y \in [y_0,y_1]} g(x_1,y) - g(x_0,y) \\
&=&  \int_{y \in [y_0,y_1]} D_2 f(x_1,y) - D_2 f(x_0,y) \\
&=&  \int_{y \in [y_0,y_1]} D_2 (f(x_1,y) - f(x_0,y)) \\
&=& (f(x_1,y_1)-f(x_0,y_1)) - (f(x_1,y_0) - f(x_0,y_0))
\end{eqnarray*}

\noindent So we have 
\[ \int_Q D_2 D_1 f = f(x_1,y_1) - f(x_1,y_0) - f(x_0,y_1) + f(x_0,y_0) = \int_Q D_1 D_2 f \]

\noindent (b) Give a proof, independent of the one given in section 6, that $D_2 D_1 f(\textbf{x}) = D_1 D_2 f(\textbf{x})$ for each $\textbf{x} \in A$.\\

Suppose for some $\textbf{x}' \in A$ it is the case that $D_1 D_2 f(\textbf{x'}) \neq D_2 D_1 f(\textbf{x'})$. First we define\\
$\epsilon = |D_2 D_1 f(\textbf{x}') - D_1 D_2 f(\textbf{x}')|$ and let $D_2 D_1 f(\textbf{x}') > D_1 D_2 f(\textbf{x}')$. The proof is similar for the case opposite inequality.

Because both $D_2 D_1 f(\textbf{x})$ and $D_1 D_2 f(\textbf{x})$ are continuous there are cubes $C_1(\textbf{x}')$ and $C_2(\textbf{x}')$ around $\textbf{}$ such that 
\begin{eqnarray*}
|D_2 D_1 f(\textbf{x}) - D_2 D_1 f(\textbf{x}') | &<& \frac{\epsilon}{3}  \quad \text{for $\textbf{x} \in C_1(\textbf{x}')$}\\
|D_1 D_2 f(\textbf{x}) - D_1 D_2 f(\textbf{x}') | &<& \frac{\epsilon}{3}  \quad \text{for $\textbf{x} \in C_2(\textbf{x}')$}\\
\end{eqnarray*}
Both inequalities are true if we only consider the cube $C(\textbf{x}') = C_1(\textbf{x}') \cap C_2(\textbf{x}')$. From the above expressions we then have 

\begin{eqnarray*}
 -\frac{\epsilon}{3} &<& D_2 D_1 f(\textbf{x}) - D_2 D_1 f(\textbf{x}')\\
 -\frac{\epsilon}{3} &<& D_1 D_2 f(\textbf{x}') - D_1 D_2 f(\textbf{x})\\
\end{eqnarray*}
We may now calculate as follows
\begin{eqnarray*}
D_2 D_1 f(\textbf{x}) - D_1 D_2 f(\textbf{x}) &=& D_2 D_1 f(\textbf{x})  - D_2 D_1 f(\textbf{x}') + D_2 D_1 f(\textbf{x}') - D_1 D_2 f(\textbf{x}') + D_1 D_2 f(\textbf{x}') - D_1 D_2 f(\textbf{x}) \\   
&>& -\frac{\epsilon}{3} + \epsilon - \frac{\epsilon}{3} = \frac{\epsilon}{3} \\
\end{eqnarray*}
So we have $D_2 D_1 f(\textbf{x}) - D_1 D_2 f(\textbf{x}) > \frac{\epsilon}{3}$ for all $\textbf{x} \in C(\textbf{x}')$. It is trivial to show that $\int_Q (g + c) = \int g + c \nu(Q)$ where $c$ is a constant and $\nu(Q)$ represents the volume of $Q$ so, we use this fact below.
\begin{eqnarray*}
\int_{C(\textbf{x})} D_1 D_2 f &<& \int_{C(\textbf{x}')} (D_2 D_1 f - \frac{\epsilon}{3}) \\
&=& \int_{C(\textbf{x}')} D_2 D_1 f - \frac{\epsilon}{3} \nu(C(\textbf{x}')) \\
&<& \int_{C(\textbf{x}')} D_2 D_1 f \quad \text{since $\epsilon, \nu(C(\textbf{x}')) > 0$ and $\epsilon$ is fixed}\\
\end{eqnarray*}

We may make the cube $C(\textbf{x}')$ smaller if $C(\textbf{x}') \not\subset A$. The inequalities above will still hold and therefore so will the result. Thus we have contradicted the established result that $\int_Q D_2 D_1 f = \int_Q D_1 D_2 f$ which completes the proof.

\end{document}
